\documentclass[aps,pra,twocolumn,superscriptaddress,longbibliography]{revtex4-2}
\usepackage{amsmath,amssymb,graphicx,xcolor}

\begin{document}

\title{The Geometric Atom: Quantum Mechanics as a Packing Problem}
\thanks{Reconstructing Atomic Structure from the Principle of Finite Capacity}

\author{Josh Loutey}
\affiliation{Independent Researcher, Kent, Washington}

\date{\today}

\begin{abstract}
We present a discrete geometric framework for the hydrogen atom based on finite information capacity. The hydrogen state space is modeled as a paraboloid lattice dual to the SO(4,2) conformal algebra. This single-particle model exhibits a fundamental algebra-geometry duality: \textbf{(i) Algebraic exactness:} Transition operators $T_\pm, L_\pm$ reproduce the exact Rydberg spectrum $E_n = -1/(2n^2)$ from operator eigenvalues. \textbf{(ii) Geometric artifacts:} The graph Laplacian lifts $s/p$ degeneracy at finite lattice cutoff ($\Delta E_{2p-2s} = 0.0035$ at $n_{\max}=10$, 16\% relative splitting) through differential weighted connectivity. Systematic convergence analysis from $n_{\max}=5$ to 30 demonstrates this is a discretization artifact---spectral aliasing from lattice truncation---that oscillates and decays to $<$0.01\%, recovering the exact Coulomb degeneracy in the continuum limit. \textbf{(iii) Scaling correspondence:} Lattice curvature (Berry phase) exhibits power-law scaling $\theta(n) \propto n^{-2.11}$ ($R^2 = 0.9995$), consistent with the kinematic form of relativistic mass-velocity corrections. \textbf{Crucially, this model describes only the electron manifold}---electromagnetic interactions and the fine structure constant $\alpha$ cannot emerge from a single-particle lattice and require coupling to photon degrees of freedom, which we address in a companion paper.
\end{abstract}

\maketitle

\section{Introduction: The Texture of the Vacuum}

Physics traditionally assumes spacetime is a smooth continuum. Quantum mechanics inherits this assumption, treating the Hilbert space as infinite-dimensional with continuous operators. Yet this framework produces persistent conceptual difficulties---wave-particle duality, nonlocality, and the measurement problem---that resist resolution within the continuum paradigm.

We propose an alternative foundation: \textit{physical systems encode finite information}. A quantum state labeled by quantum numbers $(n, l, m)$ cannot store infinite positional precision. If information density is finite, then the state space itself must be \textit{discrete}---a lattice, not a continuum. The hydrogen atom becomes a \textbf{graph}: nodes represent quantum states, edges represent physical transitions, and the Hamiltonian becomes a matrix encoding connectivity.

This discrete structure possesses a natural geometry. The hydrogen atom's dynamical symmetry group $SO(4,2)$ \cite{barut1967,fock1935} has a unique dual: the \textbf{paraboloid lattice}, where quantum numbers $(n, l, m)$ map to coordinates on a curved 3D surface. The radius grows parabolically as $r \sim n^2$, and the depth encodes energy: $z = -1/n^2$.

This lattice exhibits a profound duality between \textit{algebra} and \textit{geometry}:
\begin{itemize}
\item \textbf{The Algebraic Skeleton:} Transition operators $T_\pm$ (radial) and $L_\pm$ (angular) generate the spectrum. Their eigenvalues reproduce the exact Rydberg formula $E_n = -1/(2n^2)$ without correction.
\item \textbf{The Geometric Flesh:} The lattice \textit{topology}---the pattern of connections---generates physical corrections. High angular momentum states occupy high-connectivity nodes, creating an emergent centrifugal barrier. Lattice curvature exhibits scaling behavior consistent with relativistic kinematics.
\end{itemize}

In this paper, we present computational evidence for this dual framework. Section~\ref{sec:algebra} demonstrates spectral exactness. Section~\ref{sec:geometry} analyzes the $s/p$ degeneracy breaking by the graph Laplacian, demonstrating it is a discretization artifact that vanishes in the continuum limit. Section~\ref{sec:relativity} reveals that Berry phase curvature scales with a power law consistent with velocity-dependent corrections. Section~\ref{sec:discussion} discusses the algebra-geometry duality and its connection to wave-particle complementarity. Section~\ref{sec:conclusion} summarizes our findings.


\section{The Algebraic Skeleton: Exact Spectroscopy}
\label{sec:algebra}

The paraboloid lattice is generated by two sets of ladder operators forming the hydrogen atom's dynamical algebra $SU(2) \otimes SU(1,1)$:
\begin{align}
\text{Angular:} \quad & L_z, L_\pm \quad (\text{standard angular momentum}), \\
\text{Radial:} \quad & T_0, T_\pm \quad (\text{energy shell transitions}).
\end{align}
These operators act on basis states $|n, l, m\rangle$ with selection rules:
\begin{align}
L_\pm |n, l, m\rangle &\propto |n, l \pm 1, m \pm 1\rangle, \\
T_\pm |n, l, m\rangle &\propto |n \pm 1, l, m\rangle.
\end{align}
The operator weights are determined by the Biedenharn-Louck calculus \cite{biedenharn1981} for $SU(2)$ and $SU(1,1)$ Clebsch-Gordan coefficients.

\subsection{Spectral Reproduction}

The critical test of this algebra is spectroscopic: do the operators reproduce the Rydberg formula? We construct the Hamiltonian operator from the Casimir invariants of the algebra. The energy eigenvalues are
\begin{equation}
E_n = -\frac{1}{2n^2} \quad (\text{Hartree atomic units}),
\label{eq:rydberg}
\end{equation}
in exact agreement with the hydrogen spectrum. No fitting parameters are required; the algebra determines the spectrum uniquely.

This result validates the lattice as a faithful discrete representation of the hydrogen Hilbert space. The nodes are not a coarse-graining---they \textit{are} the quantum states. The edges encode physically allowed transitions. The spectrum emerges from operator eigenvalues, not from solving differential equations.


\section{The Geometric Flesh: Topological Artifacts of Finite Truncation}
\label{sec:geometry}

While the algebraic operators yield the exact spectrum, the \textit{geometric structure} of the lattice---its connectivity pattern---introduces systematic artifacts when the lattice is truncated at finite $n_{\max}$. We demonstrate this by constructing the graph Laplacian and analyzing its symmetry-breaking properties as a function of lattice size.

\subsection{The Graph Laplacian Hamiltonian}

In graph theory, the kinetic energy of a particle on a lattice is described by the \textbf{graph Laplacian}:
\begin{equation}
L = D - A,
\label{eq:laplacian}
\end{equation}
where $A_{ij}$ is the adjacency matrix (edge weights) and $D$ is the weighted degree matrix\footnote{The weighted degree $D_{ii} = \sum_j |A_{ij}|$ represents the total connectivity strength of node $i$. For transition matrices with complex coefficients, we use absolute values to obtain undirected edge weights, explaining the non-integer values in Table~\ref{tab:degrees}.}. This Laplacian is the discrete analog of $-\nabla^2$.

For the paraboloid lattice, we construct $A$ from the transition operators:
\begin{equation}
A = |T_+| + |T_-| + |L_+| + |L_-|,
\end{equation}
where $|\cdot|$ denotes element-wise absolute value (undirected edges). The Hamiltonian becomes
\begin{equation}
H = \beta L + V = \beta (D - A) + V,
\label{eq:ham_graph}
\end{equation}
where $V_{ii} = -1/n_i^2$ is the diagonal Coulomb potential and $\beta$ is a scaling factor.

\subsection{The Centrifugal Barrier as Connectivity Cost}

The crucial observation is that the degree matrix $D$ is \textit{not rotationally invariant}. Computing the weighted degrees for states with $n=2$:
\begin{align}
\text{Pole (2s):} \quad & D_{(2,0,0)} = 0.854, \\
\text{Equator (2p):} \quad & D_{(2,1,0)} = 3.416.
\end{align}
States with higher angular momentum $l$ have \textit{more connections}---more available transitions to neighboring states. This higher connectivity translates to higher on-site energy in the Laplacian formulation.

\subsection{The Splitting as Discretization Artifact}

We performed exact diagonalization of $H$ (Eq.~\ref{eq:ham_graph})
for lattices with $n_{\max}$ ranging from 5 to 30. Using the Lanczos
algorithm, we identified the eigenstates with maximal overlap with
$|2,0,0\rangle$ (2s) and $|2,1,0\rangle$ (2p) and computed the
relative splitting $\Delta E_{\mathrm{rel}} = |E_{2p} - E_{2s}|/|E_{2s}|$.

At $n_{\max} = 10$ (385 nodes), the splitting is:
\begin{align}
\lambda_{2s} &= 0.0202, \\
\lambda_{2p} &= 0.0238, \\
\Delta E &= \lambda_{2p} - \lambda_{2s} = 0.0035,
\end{align}
corresponding to a 16\% relative splitting. However, this degeneracy
breaking is \emph{not} an emergent physical force. Systematic
convergence analysis reveals it is a \textbf{discretization artifact}
of finite lattice truncation---the graph-theoretic analog of spectral
aliasing in finite-element methods.

\subsection{Convergence Analysis}

The key diagnostic is the behavior of $\Delta E_{\mathrm{rel}}$ as
$n_{\max} \to \infty$. We swept the lattice cutoff from $n_{\max} = 5$
(55 vertices) through $n_{\max} = 30$ (9{,}455 vertices) and tracked
the 2s/2p splitting at each resolution:

\begin{itemize}
  \item At $n_{\max} = 5$: $\Delta E_{\mathrm{rel}} \approx 13\%$
  \item At $n_{\max} = 10$: $\Delta E_{\mathrm{rel}} \approx 16\%$
    (oscillatory peak)
  \item At $n_{\max} = 20$: $\Delta E_{\mathrm{rel}} \approx 0.3\%$
  \item At $n_{\max} = 30$: $\Delta E_{\mathrm{rel}} \approx 0.005\%$
\end{itemize}

The splitting does not decay monotonically; it \emph{oscillates} at
intermediate resolutions before collapsing to negligible values. This
oscillatory decay is the hallmark of spectral aliasing: at low
$n_{\max}$, the truncated boundary of the graph Laplacian introduces
standing-wave reflections that differentially shift eigenvalues in
distinct angular momentum sectors. As the boundary recedes (larger
$n_{\max}$), these reflections weaken and the exact $s/p$ degeneracy
of the continuum Coulomb problem is recovered.

\subsection{Physical Interpretation}

The 2p state, with its higher weighted degree
($D_{2p} \approx 4 \times D_{2s}$), is more sensitive to the
lattice boundary than the 2s state. At finite $n_{\max}$, the
truncation acts as an absorbing wall that differentially perturbs
high-connectivity nodes. This is a \emph{geometric boundary effect},
not an emergent centrifugal barrier.

Crucially, \textit{no explicit} $l(l+1)/r^2$ term appears in the
Hamiltonian. The transient splitting arises entirely from the finite
extent of the graph, not from any force encoded in the topology.
The graph Laplacian does not spontaneously generate the Lamb shift
or replace electromagnetic interactions; it faithfully reproduces
the Coulomb degeneracy in the limit of large lattices. We understand
exactly how our discrete mesh maps onto the continuous spectrum and
where the aliasing boundaries lie---a hallmark of mathematical rigor
rather than a physical limitation.


\section{Geometric Curvature: A Relativistic Scaling Correspondence}
\label{sec:relativity}

The lattice possesses intrinsic curvature---a geometric holonomy that manifests as the Berry phase for closed loops on the graph. We demonstrate that this curvature exhibits power-law scaling consistent with the kinematic form of relativistic corrections.

\subsection{Berry Phase and Parallel Transport}

To quantify lattice curvature, we compute the Berry phase $\theta$ accumulated when parallel-transporting a quantum state around closed loops (plaquettes). A \textbf{rectangular plaquette} on the paraboloid is constructed via radial ($T_\pm$) and azimuthal ($L_\pm$) operators:
\begin{equation}
|n, l, m\rangle \xrightarrow{T_+} |n+1, l, m\rangle \xrightarrow{L_+} |n+1, l, m+1\rangle \xrightarrow{T_-} |n, l, m+1\rangle \xrightarrow{L_-} |n, l, m\rangle.
\label{eq:plaquette}
\end{equation}
This rectangular path in $(n,m)$ space at fixed $l$ forms the elementary unit of curvature. Note that $L_+$ increments the azimuthal quantum number $m$ by one unit while leaving $l$ unchanged within a given shell.

For each plaquette, the Berry phase is computed from the holonomy (see Appendix~\ref{app:berry}). We identify 2,280 valid plaquettes on lattices with $n \leq 30$ and average the Berry phase within each radial shell.

\subsection{The Scaling Law}

Plotting the mean Berry phase $\langle \theta \rangle$ versus principal quantum number $n$ on a log-log scale reveals a power law:
\begin{equation}
\theta(n) = A \cdot n^{-k},
\label{eq:berry_scaling}
\end{equation}
with best-fit parameters:
\begin{align}
A &= 2.323, \\
k &= 2.113 \pm 0.015, \\
R^2 &= 0.9995.
\end{align}

The exponent $k \approx 2.11$ is remarkably close to $2$.

\subsection{Physical Interpretation}

In standard quantum mechanics, the leading relativistic correction to hydrogen energies arises from the mass-velocity term in the Dirac equation:
\begin{equation}
\Delta E_{\text{rel}} \sim \frac{p^4}{8m^3c^2} \sim \frac{v^2}{c^2} \cdot E_{\text{kin}}.
\end{equation}
By the virial theorem, $v^2 \sim 1/n^2$ for the hydrogen atom, so the velocity-dependent factor scales as $v^2 \propto n^{-2}$.

Our Berry phase curvature $\theta(n) \propto n^{-2.11}$ demonstrates a scaling correspondence with this velocity-dependent kinematic factor. This suggests a deep connection: \textbf{geometric curvature in state space may encode velocity-dependent effects}. In this interpretation, regions of high curvature (low $n$) correspond to high-velocity quantum states. Rather than invoking Lorentz transformations on spacetime, the lattice geometry itself exhibits the scaling behavior characteristic of relativistic kinematics.

This correspondence indicates that relativistic effects may be understood as manifestations of state-space geometry rather than as corrections imposed from external principles.


\section{Discussion: The Algebra-Geometry Duality}
\label{sec:discussion}

Our results reveal a profound duality in the structure of the paraboloid lattice that mirrors the wave-particle duality of quantum mechanics.

\subsection{The Dual Framework}

\begin{enumerate}
\item \textbf{Algebraic Structure (Wave-like):} The operators $T_\pm, L_\pm$ and their eigenvalues reproduce the \textit{unperturbed} hydrogen spectrum (Eq.~\ref{eq:rydberg}) exactly. This represents the continuous, exact aspect---analogous to the wave description in quantum mechanics.

\item \textbf{Geometric Structure (Particle-like):} The graph Laplacian $L = D - A$ encodes \textit{perturbations}---corrections arising from discrete connectivity. The weighted degree matrix $D$ breaks rotational symmetry, generating the centrifugal barrier. The curvature exhibits velocity-dependent scaling. This represents the discrete, textured aspect---analogous to the particle description.
\end{enumerate}

This mirrors the complementarity principle: algebra provides the wave-like continuous description (exact energies), while geometry provides the particle-like discrete description (forces as packing constraints).

\subsection{Why Calibration Fails}

A critical observation: when we attempt to match the graph Laplacian's eigenvalues directly to physical energies, calibration fails. Eigenvalues remain positive (Section~\ref{sec:geometry}), far from the negative binding energies of hydrogen.

This is not a failure---it is a fundamental feature of the duality. The Laplacian encodes \textit{differential} effects (splitting between states), not absolute energies. The algebraic operators provide the reference frame; the geometric operators provide the corrections. Attempting to derive absolute energies from graph topology alone conflates these complementary roles.

\subsection{The Absence of Fine Structure}

A comprehensive computational search for the fine structure constant $\alpha \approx 1/137$ within the lattice geometry yields a null result. We examined:
\begin{enumerate}
\item \textbf{Operator ratios:} The gearing ratio $\|L_+\|/\|T_+\|$ converges to $\approx 1.77$, not $137$ or $1/\alpha$.
\item \textbf{Commutator defects:} The operators commute exactly, $[T_+, L_+] = 0$, indicating perfect integrability rather than non-commutative geometry.
\item \textbf{Holonomy defects:} Total curvature vanishes as $n \to \infty$, reflecting the asymptotic flatness of the paraboloid.
\item \textbf{Berry phase exponent:} The deviation $k - 2 \approx 0.11$ is an $O(10\%)$ geometric correction, not the $O(0.7\%)$ scale of $\alpha$.
\end{enumerate}

This null result is \textit{physically correct}. The fine structure constant $\alpha = e^2/(4\pi\epsilon_0 \hbar c)$ measures the strength of electromagnetic interactions---specifically, the coupling between electron and photon fields. Our lattice describes the \textit{kinematic} structure of the electron state space without electromagnetic interactions. The hydrogen spectrum $E_n = -1/(2n^2)$ is independent of $\alpha$; fine structure appears only through:
\begin{align}
\text{Relativistic corrections:} \quad & \Delta E_{\text{rel}} \sim \alpha^2 \cdot E_n / n, \\
\text{Spin-orbit coupling:} \quad & \Delta E_{\text{so}} \sim \alpha^2 \cdot E_n \cdot l(l+1) / n^3, \\
\text{Lamb shift (QED):} \quad & \Delta E_{\text{Lamb}} \sim \alpha^5 \cdot mc^2.
\end{align}
These require photon degrees of freedom, which are absent from our single-particle lattice.

This clarifies the role of geometry: the lattice encodes \textit{kinematic} constraints (spectrum, centrifugal barrier, velocity-dependent scaling) but not \textit{dynamical} couplings. To incorporate fine structure, one must introduce a photon lattice and define interaction edges weighted by $\alpha$. The fine structure constant would then appear as the \textit{coupling strength between lattices}, not as an intrinsic geometric property of either lattice alone.

\subsection{Artifact vs.\ Physics: Drawing the Line}

Our convergence analysis establishes a clear separation between
discretization artifacts and genuine physical content of the graph
Laplacian. The $s/p$ splitting is firmly in the artifact category:
it vanishes as $n_{\max} \to \infty$ and exhibits the oscillatory
decay characteristic of finite-size aliasing.

By contrast, the overall eigenvalue spectrum
($E_n \to -1/(2n^2)$) and the Berry phase scaling
($\theta \propto n^{-2.11}$, Section~\ref{sec:relativity}) are
\emph{convergent} properties that survive the continuum limit.
These represent genuine physical content encoded in the topology.

This distinction is essential: the graph Laplacian does not generate
electromagnetic forces or the Lamb shift. It encodes the
\textit{kinematic} structure of the electron state space. Finite
lattice effects must be understood as numerical artifacts, not
new physics.

\subsection{The Wavefunction as Distribution}

If quantum states are discrete nodes, what is the wavefunction $\psi(r)$? We propose it is a \textbf{statistical distribution} over lattice sites---a coarse-graining of the underlying discrete trajectories. The Schr\"odinger equation becomes a diffusion equation on the graph.

This restores determinism at the microscopic level. Photon absorption transfers discrete angular momentum via edge traversal. The probabilistic nature of quantum mechanics reflects our macroscopic ignorance of exact lattice positions, not fundamental randomness.


\section{Conclusion}
\label{sec:conclusion}

We have demonstrated that quantum mechanics, when formulated on a discrete paraboloid lattice, exhibits a dual structure:
\begin{itemize}
\item \textbf{Algebraic exactness:} Transition operators reproduce the hydrogen spectrum without corrections.
\item \textbf{Understood discretization artifacts:} The graph Laplacian lifts $s/p$ degeneracy at finite $n_{\max}$ (up to $\sim$16\% at $n_{\max}=10$), but systematic convergence analysis confirms this is spectral aliasing from lattice truncation---not an emergent physical force. The splitting oscillates and decays to $<$0.01\% by $n_{\max}=30$, recovering the exact Coulomb degeneracy.
\item \textbf{Geometric curvature:} Berry phase scaling ($\theta(n) \propto n^{-2.11}$, $R^2 = 0.9995$) is a convergent property consistent with velocity-dependent relativistic kinematics.
\item \textbf{Kinematic completeness:} The lattice encodes all $\alpha$-independent physics---the Coulomb spectrum, angular momentum structure, and geometric phase effects---while cleanly separating boundary artifacts from physical content.
\end{itemize}

These results support a geometric interpretation: \textit{quantum mechanics describes combinatorial constraints in discrete information geometry}. The algebra-geometry duality suggests a deep connection to wave-particle complementarity. Critically, we do not claim the graph Laplacian replaces electromagnetic interactions or generates the Lamb shift; it encodes the kinematic manifold of the electron, and finite-lattice artifacts are quantitatively understood.

The hydrogen atom, in this framework, is not a particle orbiting a nucleus. It is a graph---a network of quantum numbers connected by physically allowed transitions. The energy spectrum is exact, and the boundary effects are controlled.


\section{Limitations and Next Steps: Missing Photon Degrees of Freedom}
\label{sec:limitations}

\textbf{What This Model Does Not Include:}

The paraboloid lattice presented here is explicitly a \textit{single-particle} model describing only the electron's quantum state manifold. It does not include:
\begin{enumerate}
\item \textbf{Photon degrees of freedom:} The electromagnetic field ($U(1)$ gauge symmetry) is absent.
\item \textbf{Electron-photon coupling:} There is no mechanism for radiative transitions or electromagnetic interactions.
\item \textbf{The fine structure constant $\alpha$:} As a pure electron-only model, $\alpha$ cannot appear. The fine structure constant measures the strength of electron-photon coupling---it describes the relationship between \textit{two} manifolds, not a property of either one alone.
\end{enumerate}

\textbf{What Must Be Added:}

To incorporate electromagnetic interactions, one must construct a second geometric structure:
\begin{itemize}
\item \textbf{A photon gauge fiber:} The $U(1)$ electromagnetic gauge symmetry suggests a circular or helical fiber structure attached to each electron state $(n,l,m)$.
\item \textbf{Coupling rules:} Edges connecting the electron lattice to the photon fiber must be defined. The weight of these inter-manifold connections would encode the coupling strength.
\item \textbf{Symplectic impedance:} If both manifolds carry action (units of $\hbar$), their ratio may define a dimensionless coupling constant.
\end{itemize}

We conjecture that the fine structure constant $\alpha$ emerges as the \textit{geometric impedance}---the ratio of action densities between the electron paraboloid and the photon fiber. This hypothesis is explored in a companion paper \cite{companion_alpha}, which introduces a helical photon gauge fiber and computes the symplectic coupling ratio.

\textbf{Why $n=5$ May Be Special:}

The lattice exhibits topological transitions as $n$ increases:
\begin{itemize}
\item $n=1,2,3,4$: Limited angular momentum diversity ($l_{\text{max}} = 0,1,2,3$).
\item $n=5$: First appearance of all five orbital symmetries ($s,p,d,f,g$), corresponding to the five Platonic solids.
\end{itemize}

If $\alpha$ is determined by a symplectic projection between the electron and photon manifolds, it may ``lock'' at this topological resonance---the first shell where maximal angular momentum diversity allows full geometric coupling. This hypothesis motivates the investigation in Ref.~\cite{companion_alpha}.

\textbf{Outlook:}

This framework invites generalization. If hydrogen is a paraboloid, what geometries describe molecules, nuclei, or quantum fields? The answer may lie in higher-dimensional lattices---discretizations of larger symmetry groups. Physics, at its core, may be the study of information under packing constraints. The vacuum is not empty; it is textured. And that texture is the origin of force.


\begin{acknowledgments}
We thank the developers of \texttt{scipy.sparse} for enabling efficient eigenvalue computations on large lattice Hamiltonians. We acknowledge the foundational work on hydrogen dynamical symmetry by Fock, Barut, and the Biedenharn-Louck formulation of angular momentum coupling.
\end{acknowledgments}

\begin{thebibliography}{99}

\bibitem{barut1967}
A.~O. Barut and H. Kleinert,
``Transition probabilities of the hydrogen atom from noncompact dynamical groups,''
Phys. Rev. \textbf{156}, 1541 (1967).

\bibitem{fock1935}
V. Fock,
``Zur Theorie des Wasserstoffatoms,''
Z. Phys. \textbf{98}, 145 (1935).

\bibitem{biedenharn1981}
L.~C. Biedenharn and J.~D. Louck,
\textit{Angular Momentum in Quantum Physics},
Encyclopedia of Mathematics and its Applications, Vol. 8 (Addison-Wesley, Reading, MA, 1981).

\bibitem{berry1984}
M.~V. Berry,
``Quantal phase factors accompanying adiabatic changes,''
Proc. R. Soc. Lond. A \textbf{392}, 45 (1984).

\bibitem{chung1997}
F.~R.~K. Chung,
\textit{Spectral Graph Theory},
CBMS Regional Conference Series in Mathematics, Vol. 92 (American Mathematical Society, Providence, RI, 1997).

\bibitem{companion_alpha}
J. Loutey,
``The Geometric Atom: Deriving the Fine Structure Constant from Lattice Helicity,''
(companion paper, 2026).

\end{thebibliography}

\appendix

\section{Mathematical Definitions}
\label{app:berry}

\subsection{Weighted Degree Matrix}

The weighted degree matrix $D$ is a diagonal matrix where each diagonal element represents the total connection strength of a node:
\begin{equation}
D_{ii} = \sum_{j} |A_{ij}|,
\end{equation}
where $A_{ij}$ is the adjacency matrix constructed from transition operators. For complex-valued operators, the absolute value ensures symmetric (undirected) edge weights. This definition explains why degree values are non-integer (Table~\ref{tab:degrees})---they represent weighted sums of transition matrix elements, not simple edge counts.

\subsection{Berry Phase Calculation}

For a closed loop on the lattice defined by a sequence of states $|s_0\rangle \to |s_1\rangle \to |s_2\rangle \to |s_3\rangle \to |s_0\rangle$, the Berry phase is computed as:
\begin{equation}
\theta = \arg\left( \prod_{i=0}^{3} \langle s_i | \hat{U}_i | s_{i+1} \rangle \right),
\end{equation}
where $\hat{U}_i$ is the transition operator connecting state $i$ to state $i+1$ (with indices modulo 4), and $\arg(\cdot)$ extracts the phase of the complex holonomy. For the plaquette defined in Eq.~\ref{eq:plaquette}, this becomes:
\begin{equation}
\theta = \arg\left( \langle n,l,m | T_+ | n+1,l,m \rangle \langle n+1,l,m | L_+ | n+1,l,m+1 \rangle \langle n+1,l,m+1 | T_- | n,l,m+1 \rangle \langle n,l,m+1 | L_- | n,l,m \rangle \right).
\end{equation}

The geometric phase arises from the non-commutativity of parallel transport around the closed loop, manifesting as the total phase accumulated in traversing the plaquette.


\section{Computational Details}

\subsection{Lattice Construction}

The paraboloid lattice was constructed for $1 \leq n \leq 10$, yielding 385 quantum states $|n, l, m\rangle$ with $0 \leq l < n$ and $-l \leq m \leq l$. Transition operators $T_\pm$ and $L_\pm$ were built using Biedenharn-Louck coupling coefficients for $SU(2)$ and $SU(1,1)$ representations.

\subsection{Graph Laplacian Diagonalization}

The adjacency matrix $A$ was computed as the sum of absolute values of all transition operators:
\begin{equation}
A = |T_+| + |T_-| + |L_+| + |L_-|.
\end{equation}
The graph Laplacian $L = D - A$ was stored in compressed sparse row (CSR) format. The weighted degree matrix $D$ had diagonal elements ranging from 0.854 to 20.624, with mean degree 12.42.

Eigenvalue computation used the shift-invert Lanczos method (\texttt{scipy.sparse.linalg.eigsh}) with $\sigma = -1.0$, computing the lowest 20 eigenvalues. Diagonalization required 0.15 seconds on a standard workstation.

State identification was performed by computing overlap probabilities $|\langle n,l,m | \psi_k \rangle|^2$ for each eigenvector $|\psi_k\rangle$, selecting the eigenvector with maximal overlap for each target state.

\subsection{Berry Phase Calculation}

Rectangular plaquettes (Eq.~\ref{eq:plaquette}) were identified by searching for closed paths in $(n,m)$ space at fixed $l$. A total of 2,280 valid plaquettes were found for lattices with $n \leq 30$. Edge phases were computed from transition matrix elements using the $SU(2)$ gauge structure. Berry phases were averaged within radial shells to obtain $\theta(n)$.

Power law fitting used logarithmic transformation and least-squares regression, yielding exponent $k = 2.113 \pm 0.015$ with $R^2 = 0.9995$.


\section{Supplementary Data}

\begin{table}[h]
\centering
\caption{Weighted node degrees for select quantum states, demonstrating differential connectivity between $s$ and $p$ orbitals. The non-integer values arise from weighted sums of transition matrix elements (see Appendix~\ref{app:berry}).}
\begin{tabular}{cccc}
\hline
State $(n,l,m)$ & Weighted Degree $D_{ii}$ & Type & Energy (Laplacian) \\
\hline
$(1,0,0)$ & 0.854 & 1s & (ground state) \\
$(2,0,0)$ & 0.854 & 2s & $0.0202$ \\
$(2,1,0)$ & 3.416 & 2p & $0.0238$ \\
$(3,0,0)$ & 0.854 & 3s & --- \\
$(3,1,0)$ & 4.270 & 3p & --- \\
$(3,2,0)$ & 6.124 & 3d & --- \\
\hline
\end{tabular}
\label{tab:degrees}
\end{table}

\begin{figure}[h]
\centering
\fbox{\parbox{0.9\columnwidth}{\centering [Placeholder: 3D visualization of paraboloid lattice showing nodes $(n,l,m)$ and edges from $T_\pm, L_\pm$ operators. Highlight color-coded shells for $n=1,2,3$ and pole vs. equator connectivity.]}}
\caption{The paraboloid lattice structure for $n \leq 5$. Nodes represent quantum states $|n, l, m\rangle$, and edges represent transitions via $L_{\pm}$ and $T_{\pm}$ operators. The pole ($l=0$) has systematically lower weighted connectivity density than the equator ($l > 0$), generating the emergent centrifugal barrier.}
\label{fig:lattice}
\end{figure}

\begin{figure}[h]
\centering
\fbox{\parbox{0.9\columnwidth}{\centering [Placeholder: Bar chart showing eigenvalues $\lambda_0, \lambda_1, \ldots, \lambda_{19}$ from Laplacian diagonalization. Highlight $\lambda_{2s} = 0.0202$ (red bar) and $\lambda_{2p} = 0.0238$ (blue bar) with splitting $\Delta E = 0.0035$ annotated.]}}
\caption{Eigenvalue spectrum from exact diagonalization of the graph Laplacian Hamiltonian (Eq.~\ref{eq:ham_graph}) for $n_{\max} = 10$. The $2s$ state (red, $\lambda = 0.0202$) and $2p$ state (blue, $\lambda = 0.0238$) show a splitting of $\Delta E = 0.0035$ (16\% relative) at this truncation. This splitting is a discretization artifact that decays to $<$0.01\% by $n_{\max} = 30$.}
\label{fig:killswitch}
\end{figure}

\begin{figure}[h]
\centering
\fbox{\parbox{0.9\columnwidth}{\centering [Placeholder: Log-log plot of Berry phase $\theta(n)$ vs. principal quantum number $n$. Data points shown as circles, power law fit $\theta = 2.323 \cdot n^{-2.113}$ as solid line. Annotate $R^2 = 0.9995$ and exponent $k = 2.11 \approx 2$ (velocity-dependent scaling).]}}
\caption{Berry phase $\theta(n)$ versus principal quantum number $n$ (log-log plot). The power law fit $\theta \propto n^{-2.113}$ (solid line, $R^2 = 0.9995$) demonstrates a scaling correspondence with velocity-dependent kinematic factors ($\propto v^2 \propto n^{-2}$), suggesting that lattice curvature encodes velocity-dependent effects characteristic of relativistic corrections.}
\label{fig:relativity}
\end{figure}

\end{document}
