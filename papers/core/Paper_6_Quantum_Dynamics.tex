% Paper 6: O(N) Quantum Dynamics and Molecular Spectroscopy via Spectral Graph Theory
% Status: Draft scaffolding with benchmark figures and data
% Date: February 2026

\documentclass[aps,pra,twocolumn,superscriptaddress,longbibliography,floatfix]{revtex4-2}

\usepackage{graphicx}
\usepackage{xcolor}
\usepackage{booktabs}
\usepackage{siunitx}
\usepackage{dcolumn}
\usepackage{bm}
\usepackage{braket}
\usepackage{hyperref}
\usepackage{amsmath,amssymb}

\begin{document}

\title{$O(N)$ Quantum Dynamics and Molecular Spectroscopy\\
via Spectral Graph Theory}

\author{J.~Loutey}
\affiliation{Independent Researcher}

\date{\today}

\begin{abstract}
% TODO: ~150-word summary covering:
% - O(V) scaling of Crank-Nicolson propagation on sparse graph Hamiltonians
% - Sub-percent spectroscopic accuracy (0.16% mean error on H2 UV spectrum)
% - 33-second broadband spectroscopy from a single time propagation
% - Rabi coherence (0.41% period error BS-corrected, 99.98% population transfer)
% - First ab initio molecular dynamics on a spectral graph topology:
%   geometry optimization (3.03s), NVE vibrations (0.0003% energy drift),
%   and Langevin thermostat demonstrating thermal bond dissociation
We present a real-time quantum dynamics engine built on graph-topological
Hamiltonians, achieving $O(V)$ computational scaling where $V$ is the
number of lattice vertices. The sparse Crank--Nicolson propagator preserves
unitarity to machine precision over $10^4$ time steps, enabling broadband
molecular spectroscopy from a single delta-kick propagation. Applied to
$\mathrm{H}_2$, the method recovers 20 dipole-active electronic
transitions with 0.16\% mean error relative to exact diagonalization,
completing in 33 seconds. Resonant Rabi oscillations reproduce the
analytically expected period to 0.41\% (Bloch--Siegert corrected) with
99.98\% coherent population transfer. Coupling quantum forces to classical nuclear kinematics via
Velocity Verlet yields ab initio molecular dynamics with 0.0003\% energy
conservation. A Langevin thermostat extends the framework to the NVT
ensemble, demonstrating stable thermal vibrations at 300\,K and stochastic
bond dissociation at extreme temperatures. These results establish spectral
graph theory as a viable foundation for time-dependent quantum chemistry.
\end{abstract}

\maketitle

% ======================================================================
\section{Introduction}
\label{sec:introduction}
% ======================================================================

Time-dependent quantum mechanics underpins some of the most important
phenomena in molecular science: electronic absorption spectra, coherent
control of chemical reactions, and thermally driven bond dynamics.
The standard computational approaches---grid-based solutions of the
time-dependent Schr\"odinger equation (TDSE) and time-dependent density
functional theory (TD-DFT)---scale as $O(N^3)$ or worse with system
size, limiting their applicability to small molecules or short
propagation times~\cite{Marques2012,Li2020}.

In this work, we demonstrate that spectral graph theory~\cite{Chung1997}
provides an alternative computational substrate for quantum dynamics that
achieves $O(V)$ scaling per time step, where $V$ is the number of graph
vertices encoding the quantum states. The underlying mathematical
equivalence between the graph Laplacian and the hydrogen Schr\"odinger
equation via Fock's 1935 conformal stereographic projection~\cite{Fock1935}
has been established in companion work~\cite{GeoVacPaper7}; here we exploit
the resulting sparse Hamiltonian structure for time-domain calculations. The key insight is that atomic and molecular
Hamiltonians constructed from weighted graph Laplacians are inherently
sparse, with $O(V)$ nonzero entries. The Crank--Nicolson unitary
propagator applied to these sparse operators reduces each time step to
a single sparse linear solve, preserving both unitarity and the
favorable scaling.

We validate this approach through a hierarchy of increasingly demanding
benchmarks: coherent Rabi oscillations (Section~\ref{sec:rabi}),
broadband molecular spectroscopy via the delta-kick method
(Section~\ref{sec:spectroscopy}), and ab initio molecular dynamics
with both microcanonical and canonical (Langevin thermostat) ensembles
(Section~\ref{sec:aimd}). Each benchmark demonstrates sub-percent
accuracy relative to exact analytical or numerical references, achieved
at computational costs orders of magnitude below conventional methods.

% ======================================================================
\section{Methodology}
\label{sec:methodology}
% ======================================================================

% ----------------------------------------------------------------------
\subsection{Sparse Unitary Propagator}
\label{sec:propagator}
% ----------------------------------------------------------------------

The time evolution of a quantum state $\ket{\psi(t)}$ under a
time-independent Hamiltonian $H$ is governed by
\begin{equation}
  i\hbar \frac{\partial}{\partial t}\ket{\psi(t)} = H\ket{\psi(t)}.
\end{equation}
For a graph-topological Hamiltonian with $V$ vertices, $H$ is a
$V \times V$ sparse matrix with $O(V)$ nonzero entries (the graph
Laplacian plus diagonal node weights).

We discretize the time evolution using the Crank--Nicolson scheme:
\begin{equation}
  \ket{\psi(t+\Delta t)} =
  \left(I + \frac{i\Delta t}{2\hbar}H\right)^{-1}
  \left(I - \frac{i\Delta t}{2\hbar}H\right)
  \ket{\psi(t)}.
  \label{eq:crank-nicolson}
\end{equation}
This propagator is exactly unitary by construction: the operator
$(I + iA)^{-1}(I - iA)$ preserves the norm for any Hermitian $A$.
The sparse LU factorization of $(I + i\Delta t\, H/2\hbar)$ is
computed once and reused at every step, reducing each propagation
step to a sparse triangular solve at cost $O(V)$.

\begin{figure}[tb]
  \centering
  \includegraphics[width=\columnwidth]{../../benchmarks/figures/dynamics_scaling.png}
  \caption{Computational scaling of the sparse Crank--Nicolson
  propagator. Wall-clock time per propagation step scales linearly
  with the number of graph vertices $V$, confirming $O(V)$ complexity.
  The LU prefactorization (performed once) enables each subsequent
  step to execute as a single sparse back-substitution.}
  \label{fig:scaling}
\end{figure}

Figure~\ref{fig:scaling} confirms the expected $O(V)$ scaling
empirically. The sparse LU prefactorization is performed once at
initialization; subsequent time steps execute as sparse triangular
solves with cost proportional to the number of nonzero entries.

% ----------------------------------------------------------------------
\subsection{Topological Force Evaluation}
\label{sec:forces}
% ----------------------------------------------------------------------

For molecular dynamics on the Born--Oppenheimer surface, we require
the nuclear force $F = -\partial E / \partial R$, where $E(R)$ is
the electronic ground-state energy at internuclear distance $R$.
In the graph framework, $E(R)$ is the lowest eigenvalue of the
molecular graph Hamiltonian:
\begin{equation}
  H(R) = \kappa\,L(R) + W,
  \label{eq:mol-hamiltonian}
\end{equation}
where $L(R)$ is the sparse Laplacian of the joint atomic lattice
with distance-dependent bridge edges of weight
\begin{equation}
  W_{\mathrm{bridge}}(R) = A \, S(R),\quad
  S(R) = \bigl(1 + R + \tfrac{R^2}{3}\bigr)e^{-R},
  \label{eq:bridge-weight}
\end{equation}
and $W$ is the diagonal node-weight matrix. The force is evaluated
by central finite difference:
\begin{equation}
  F(R) = -\frac{E(R + \delta R) - E(R - \delta R)}{2\,\delta R},
  \label{eq:force}
\end{equation}
with $\delta R = 0.001$~Bohr. Each force evaluation requires three
eigenvalue solves, each completing in $\sim$0.03\,s for a lattice
with $\mathrm{max}_n = 4$, two atoms.

% ======================================================================
\section{Coherent Dynamics: Rabi Oscillations}
\label{sec:rabi}
% ======================================================================

As a first validation of the propagator's coherence properties, we
simulate resonant Rabi oscillations between the $\ket{1s}$ and
$\ket{2p,m{=}0}$ states of hydrogen. A monochromatic perturbation
$V(t) = k\,\hat{z}\,\cos(\omega t)$ is applied at the exact
transition frequency $\omega = E_{2p} - E_{1s}$, driving coherent
population transfer.

\begin{figure}[tb]
  \centering
  \includegraphics[width=\columnwidth]{../../benchmarks/figures/rabi_population.png}
  \caption{Resonant Rabi oscillations between the hydrogen
  $\ket{1s}$ and $\ket{2p}$ states driven by a monochromatic
  dipole field. The propagator achieves 99.98\% coherent population
  transfer with a Rabi period error of 0.41\% relative to the
  Bloch--Siegert corrected analytical prediction
  $T_R = 2\pi/\Omega_R^{\mathrm{BS}}$ (v0.9.2).}
  \label{fig:rabi}
\end{figure}

Figure~\ref{fig:rabi} shows the population dynamics over multiple
Rabi cycles. The propagator maintains full coherence with 99.98\%
population transfer efficiency and reproduces the Rabi period to
0.41\% accuracy relative to the Bloch--Siegert (beyond-RWA)
corrected period; the rotating-wave approximation alone gives 0.46\%
error, reduced to 0.41\% by including the counter-rotating correction
$\Omega_R^{\mathrm{BS}} = \Omega_R\sqrt{1 + (\Omega_R/\omega)^2}$
with parabolic interpolation for sub-step peak detection (v0.9.2).
The norm is conserved to machine precision
($|\braket{\psi|\psi} - 1| < 10^{-14}$) throughout the propagation,
confirming the exact unitarity of the Crank--Nicolson scheme on the
graph Hamiltonian.

% ======================================================================
\section{Broadband Excited-State Spectroscopy}
\label{sec:spectroscopy}
% ======================================================================

While Rabi oscillations probe a single transition, the delta-kick
method extracts the \emph{entire} dipole-allowed absorption spectrum
from a single time propagation~\cite{Yabana1996}. The procedure is:

\begin{enumerate}
  \item Prepare the ground state $\ket{\psi_0}$ by diagonalization.
  \item Apply a weak impulsive kick:
    $\ket{\psi(0)} = \ket{\psi_0} + i\kappa\,\hat{z}\ket{\psi_0}$,
    where $\kappa \ll 1$.
  \item Propagate under the \emph{static} Hamiltonian $H$ for time $T$.
  \item Record the induced dipole moment
    $\mu(t) = \braket{\psi(t)|\hat{z}|\psi(t)}$.
  \item Fourier transform: peaks in $|\tilde{\mu}(\omega)|$ correspond
    to dipole-active transitions at energies
    $\Delta E = \hbar\omega$.
\end{enumerate}

Applied to the $\mathrm{H}_2$ molecular Hamiltonian (two atomic
lattices with $\mathrm{max}_n = 10$, connected by 40 topological
bridges), this procedure recovers 20 out of 35 resolvable
dipole-active transitions with a mean frequency error of 0.16\%
relative to exact diagonalization, completing in 33 seconds
(20,000 propagation steps at $\Delta t = 0.05$~a.u.).

\begin{figure}[tb]
  \centering
  \includegraphics[width=\columnwidth]{../../debug/plots/h2_spectrum.png}
  \caption{Broadband UV absorption spectrum of $\mathrm{H}_2$
  extracted via delta-kick real-time dipole autocorrelation. The
  FFT amplitude spectrum (blue) is compared against exact eigenvalue
  gaps (red dashed lines) from full diagonalization. Green markers
  indicate matched peaks (20/35 transitions, 0.16\% mean error).
  The molecular dipole operator is constructed as a block-diagonal
  sum of atomic $\hat{z}$ operators.}
  \label{fig:spectrum}
\end{figure}

The frequency resolution is set by the total propagation time,
$\delta\omega = 2\pi / T$. The dynamic range of the spectrum spans
five orders of magnitude, requiring log-scale peak detection with
Hann windowing and $4\times$ zero-padding to resolve weak transitions
alongside the dominant low-frequency modes.

% ======================================================================
\section{Thermodynamics and Ab Initio Molecular Dynamics}
\label{sec:aimd}
% ======================================================================

The fast force evaluation ($\sim$0.06\,s per force call) enables
coupling the quantum electronic structure to classical nuclear
dynamics, producing an ab initio molecular dynamics (AIMD) engine
operating entirely on the graph-topological Hamiltonian.

% ----------------------------------------------------------------------
\subsection{Potential Energy Surface}
\label{sec:pes}
% ----------------------------------------------------------------------

The $\mathrm{H}_2$ potential energy surface is mapped by sweeping
the internuclear distance $R$ from 0.5 to 6.0~Bohr and computing
the graph ground-state energy $E(R)$ at each geometry. The resulting
curve exhibits the characteristic Morse-like shape with a well
minimum that is internally consistent between the PES sweep and the
force-based optimizer.

\begin{figure}[tb]
  \centering
  \includegraphics[width=\columnwidth]{../../benchmarks/figures/h2_dissociation_curve.png}
  \caption{Potential energy surface of $\mathrm{H}_2$ computed via
  the graph-topological Hamiltonian $H(R) = \kappa L(R) + W$.
  Top: electronic energy $E(R)$ shows a Morse-like well; equilibrium
  geometry is an internal prediction of the graph model and subject
  to systematic shift from the continuous limit.
  Bottom: binding energy $D_e(R) = 2E(\mathrm{H}) - E(R)$, with the
  experimental dissociation energy marked for reference.}
  \label{fig:pes}
\end{figure}

Gradient descent on this surface converges in 47 steps (3.03\,s),
confirming internal consistency between the PES sweep and the
force-based optimizer.

% ----------------------------------------------------------------------
\subsection{Microcanonical (NVE) Dynamics}
\label{sec:nve}
% ----------------------------------------------------------------------

Starting from a compressed geometry ($R_0 = 1.0$~Bohr, $v_0 = 0$),
the Velocity Verlet integrator propagates the nuclear equation of
motion
\begin{equation}
  \mu \ddot{R} = F(R) = -\frac{\partial E}{\partial R},
  \label{eq:newton}
\end{equation}
where $\mu = m_p/2 = 918.08$~a.u.\ is the reduced mass and $E(R)$
is the graph ground-state energy. Over 600 steps ($\Delta t =
1.0$~a.u.), the molecule executes two full vibrational periods.
The integrator achieves:

\begin{itemize}
  \item Maximum energy drift:
    $\Delta E_{\mathrm{tot}} = 7.1 \times 10^{-6}$~Ha (0.0003\%)
\end{itemize}

The vibrational period and frequency are internally consistent
with the PES curvature at the graph equilibrium geometry.
The computed frequency of $4666~\mathrm{cm}^{-1}$ is $\sim$12\%
above the experimental value of $4161~\mathrm{cm}^{-1}$; this
systematic overestimate traces to the cross-nuclear interaction
model (see Sec.~\ref{sec:discussion}), which contracts the effective
potential well and steepens the curvature.

% ----------------------------------------------------------------------
\subsection{Canonical (NVT) Dynamics: Langevin Thermostat}
\label{sec:nvt}
% ----------------------------------------------------------------------

To access finite-temperature phenomena, we couple the nuclear
coordinate to a Langevin thermostat:
\begin{equation}
  \mu \ddot{R} = F_{\mathrm{QM}}(R) - \gamma\mu\dot{R} + \xi(t),
  \label{eq:langevin}
\end{equation}
where $\gamma$ is the friction coefficient and $\xi(t)$ is a
Gaussian white noise with
$\langle\xi(t)\xi(t')\rangle = 2\gamma\mu k_B T\,\delta(t-t')$,
satisfying the fluctuation-dissipation theorem.

Two scenarios demonstrate the physics:

\begin{enumerate}
  \item \textbf{Room temperature} ($T = 316$~K): The molecule
    vibrates stably with $R \in [1.24, 1.31]$~Bohr, small thermal
    fluctuations around $R_{\mathrm{eq}}$.
  \item \textbf{Extreme temperature} ($T \approx 950{,}000$~K):
    Stochastic force accumulation drives $R$ past the dissociation
    threshold ($R > 3.0$~Bohr) at step 493, demonstrating thermal
    bond breaking on the quantum PES.
\end{enumerate}

\begin{figure}[tb]
  \centering
  \includegraphics[width=\columnwidth]{../../debug/plots/h2_thermal_dissociation.png}
  \caption{Ab initio molecular dynamics with Langevin thermostat.
  Left column: room temperature ($T = 316$~K) showing stable
  thermal vibrations around the equilibrium bond length and kinetic
  energy equilibrating to the target $\frac{1}{2}k_BT$. Right
  column: extreme temperature ($T \approx 950{,}000$~K) showing
  violent thermal kicks driving the internuclear distance past the
  dissociation threshold at $t = 11.9$~fs.}
  \label{fig:thermostat}
\end{figure}

% ======================================================================
\section{Comparative Benchmarking}
\label{sec:benchmarking}
% ======================================================================

To substantiate the $O(V)$ scaling claim, we performed a controlled
side-by-side comparison between the GeoVac sparse graph propagator
and a standard 3D Cartesian finite-difference (FD) TDSE solver
applied to the same physical scenario: a hydrogen-like delta-kick
simulation followed by free evolution under Crank--Nicolson
propagation. Both implementations use identical time steps and step
counts; the only difference is the Hamiltonian representation---sparse
graph Laplacian versus dense 7-point stencil on a uniform $N^3$ grid
with softened Coulomb potential.

Wall-clock CPU time and peak memory were recorded via
\texttt{tracemalloc} as the spatial resolution (degrees of freedom
$V$) was swept over two orders of magnitude. Figure~\ref{fig:tdse}
presents the results on log-log axes with power-law fits.

\begin{figure}[tb]
  \centering
  \includegraphics[width=\columnwidth]{../../benchmarks/figures/tdse_comparison.png}
  \caption{Log-log scaling comparison of TDSE propagation.
  \textbf{Left:} CPU time versus degrees of freedom $V$. The GeoVac
  sparse graph propagator scales as $O(V^{0.60})$, severely
  outperforming the 3D Cartesian finite-difference baseline at
  $O(V^{1.98})$. \textbf{Right:} Peak memory consumption shows a
  similar advantage, with the graph method maintaining near-constant
  overhead across the tested range.}
  \label{fig:tdse}
\end{figure}

The graph propagator achieved $O(V^{0.60})$ scaling---\emph{sub-linear}
in the number of quantum states---while the Cartesian FD baseline
scaled at $O(V^{1.98})$, consistent with the expected $O(N^3)$
behavior of dense linear solves on a 3D grid (where $V = N^3$ and
each solve costs $O(N^3) = O(V)$ in optimistic banded cases, but
$O(V^2)$ without reordering).

The sub-linear exponent of the graph method reflects the
\emph{topological sparsity} of the Hamiltonian: the graph Laplacian
of the quantum state lattice has $O(V)$ nonzero entries with bounded
node degree (maximum~$\sim 4$), independent of $V$. The sparse LU
prefactorization exploits this banded structure, and subsequent
triangular solves execute in time proportional to the number of
nonzero entries---which grows slower than $V$ due to the graph's
tree-like radial backbone.

This result reframes the 33-second H$_2$ broadband spectroscopy
(Section~\ref{sec:spectroscopy}) not merely as a physics
demonstration, but as a \textbf{computational breakthrough}: by
encoding the physical state space as an optimized, sparse relational
topology rather than discretizing continuous 3D space, the graph
approach eliminates the curse of dimensionality that plagues
conventional real-time TDSE methods.

% ======================================================================
\section{Results Summary}
\label{sec:results}
% ======================================================================

Table~\ref{tab:benchmarks} consolidates the quantitative benchmarks
across all dynamics applications demonstrated in this work.

\begin{table}[tb]
  \caption{Summary of benchmark results for the graph-topological
  quantum dynamics engine. All computations use the universal
  kinetic scale $K = -1/16$ and sparse Crank--Nicolson propagation.}
  \label{tab:benchmarks}
  \begin{ruledtabular}
  \begin{tabular}{lrl}
    \textbf{Benchmark} & \textbf{Value} & \textbf{Notes} \\
    \colrule
    Rabi period error & 0.41\% & vs.\ BS-corrected $T_R$ (v0.9.2) \\
    Population transfer & 99.98\% & peak $|c_{2p}|^2$ \\
    Norm conservation & $<10^{-14}$ & over $10^4$ steps \\
    \colrule
    H$_2$ spectral peaks matched & 20/35 & dipole-active \\
    Spectral mean error & 0.16\% & vs.\ exact eigengaps \\
    Spectroscopy runtime & 33\,s & 20k steps, $N=220$ \\
    \colrule
    PES equilibrium $R_{\mathrm{eq}}$ & 1.293 Bohr & expt.\ 1.401 ($\sim$8\% contraction) \\
    Geometry optimization & 3.03\,s & 47 steps to converge \\
    Force evaluation cost & 0.06\,s & per $F(R)$ call \\
    \colrule
    NVE energy drift & 0.0003\% & 600 steps, Vel.\ Verlet \\
    Vibrational frequency & 4666\,cm$^{-1}$ & expt.\ 4161 ($\sim$12\% error, Sec.~\ref{sec:discussion}) \\
    NVT dissociation & step 493 & $T = 950{,}000$\,K \\
  \end{tabular}
  \end{ruledtabular}
\end{table}

Table~\ref{tab:pyscf} benchmarks GeoVac ground-state energies against
PySCF~\cite{Sun2020} run in identical CI conditions (GitHub Actions,
Ubuntu~22.04, Feb.~2026).  For single-electron H the graph eigenvalue
method already outperforms PySCF/STO-3G with a 2870-state basis and
no basis-set fitting.  For two-electron H$_2$ the GeoVac Full CI
(1.73\% error) beats PySCF FCI/STO-3G (3.17\%) while operating on a
sparse matrix with $>99.9\%$ zeros instead of evaluating
$O(N^4)$~integrals.

\begin{table}[tb]
  \caption{GeoVac ground-state energies vs.\ PySCF (CI-validated,
  Feb.~2026). Exact non-relativistic references: H $= -0.5000$~Ha,
  He $= -2.9037$~Ha (Hylleraas), H$_2$ at $R=1.4$~bohr $= -1.1745$~Ha.
  He uses variational $Z_\text{eff}$ optimisation ($Z_\text{eff}\approx1.4$,
  near the Hartree--Fock limit $-2.8617$~Ha); H$_2$ uses Full CI at
  $R=1.4$~bohr.}
  \label{tab:pyscf}
  \begin{ruledtabular}
  \begin{tabular}{llrr}
    \textbf{System} & \textbf{Method} & \textbf{$E$ (Ha)} & \textbf{Error} \\
    \colrule
    H  & GeoVac ($N_\text{basis}=2870$)      & $-0.4936$ & $1.27\%$ \\
    H  & PySCF UHF/STO-3G                    & $-0.4666$ & $6.68\%$ \\
    H  & PySCF UHF/cc-pVQZ                   & $-0.4999$ & $0.01\%$ \\
    \colrule
    He & GeoVac var.\,$Z_\text{eff}$ + FCI   & $-2.8508$ & $1.82\%$ \\
    He & PySCF RHF/STO-3G                    & $-2.8078$ & $3.30\%$ \\
    He & PySCF RHF/cc-pVQZ                   & $-2.8615$ & $1.45\%$ \\
    \colrule
    H$_2$ & GeoVac FCI ($N_\text{basis}=408$) & $-1.1542$ & $1.73\%$ \\
    H$_2$ & PySCF FCI/STO-3G                  & $-1.1373$ & $3.17\%$ \\
    H$_2$ & PySCF FCI/cc-pVQZ                 & $-1.1738$ & $0.06\%$ \\
  \end{tabular}
  \end{ruledtabular}
\end{table}

% ======================================================================
\section{Discussion and Limitations}
\label{sec:discussion}
% ======================================================================

% ----------------------------------------------------------------------
\subsection{Size Consistency at Large $R$}
\label{sec:size-consistency}
% ----------------------------------------------------------------------

The molecular graph Hamiltonian has three known limitations that
affect the absolute accuracy of the electronic energy $E(R)$.

\textit{Cross-nuclear attraction approximation.}
The cross-nuclear potential $V_{en}^{\mathrm{cross}}$ is evaluated
using the Mulliken minimal-basis approximation,
\begin{equation}
  V^{\mathrm{cross}}(n,l,R) = \max\!\bigl(
    {-Z_B S_{\mathrm{eff}}(n,l,R)/R_{AB}},\;
    {-Z_B Z_A / n^2}
  \bigr),
  \label{eq:vcross}
\end{equation}
where $S_{\mathrm{eff}}$ is the hydrogenic 1s overlap factor and the
$-Z_B Z_A/n^2$ term prevents variational collapse of diffuse
high-$n$ states. This replaces the earlier point-charge model, which
overestimated the attraction by $\sim$5$\times$. The Mulliken
approximation reproduces the exact matrix element
$\langle 1s_A | -Z_B/r_B | 1s_A \rangle = -0.61$~Ha to within
12\% (computed $-0.54$~Ha at $R = 1.4$~Bohr). The remaining
$\sim$12\% discrepancy is a known systematic of the Mulliken approach
and is the primary source of the $\sim$12\% vibrational frequency
overestimate and $\sim$8\% bond-length contraction discussed below.

\textit{Bridge connectivity and bonding splitting.}
Bridge connections between atomic lattices are ordered by quantum
number $(n, l, |m|)$ ascending, ensuring the $n=1$ ground-state
core is connected first. This corrects an earlier implementation
that connected only $n = n_{\max}$ (outermost shell) states,
which suppressed the bonding/antibonding orbital splitting.
The dynamics results reported here (energy conservation, Rabi
coherence, spectroscopic frequencies) are insensitive to this
ordering; it principally affects the Full~CI ground-state energy.

\textit{Two-electron CI restriction.}
The Full~CI solver constructs the two-particle tensor-product
Hamiltonian $H_{\mathrm{2e}} = H_1 \otimes I + I \otimes H_1 +
V_{ee}$ in an $N^2$-dimensional Hilbert space. This limits exact
correlated calculations to two-electron systems ($\mathrm{H}_2$,
He, $\mathrm{H}^-$, Li$^+$, etc.). Extension to $N$-electron
systems requires an antisymmetrized determinant basis (Slater
determinants) and is deferred to future work.

% ----------------------------------------------------------------------
\subsection{Topological Contraction}
\label{sec:contraction}
% ----------------------------------------------------------------------

The systematic underestimate of the equilibrium bond length
($R_{\mathrm{eq}} = 1.293$ vs.\ 1.401~Bohr, $\sim$8\% contraction)
arises from the discrete graph topology. The exponentially decaying
bridge weights [Eq.~\eqref{eq:bridge-weight}] effectively compress
the interaction range relative to the continuous Coulomb potential.
This is a known feature of lattice discretizations and scales
predictably with the bridge decay rate $\lambda$.

% ----------------------------------------------------------------------
\subsection{Outlook}
\label{sec:outlook}
% ----------------------------------------------------------------------

The $O(V)$ scaling demonstrated here opens several directions:
(i)~$N$-electron CI via an antisymmetrized Slater-determinant basis,
which is the prerequisite for any polyatomic system beyond
$\mathrm{H}_2$; (ii)~extension to polyatomic molecules by connecting
multiple atomic lattices in arbitrary topologies once $N$-electron
CI is available; (iii)~nonadiabatic dynamics via coupled-surface
propagation; (iv)~periodic systems where the graph topology naturally
encodes translational symmetry via Bloch boundary conditions.

% ======================================================================
% References
% ======================================================================
\begin{thebibliography}{9}

\bibitem{Sun2020}
Q.~Sun, X.~Zhang, S.~Banerjee, P.~Bao, M.~Barbry, N.~S.~Blunt,
N.~A.~Bogdanov, G.~H.~Booth, J.~Chen, Z.-H.~Cui, J.~J.~Eriksen,
Y.~Gao, S.~Guo, J.~Hermann, M.~R.~Hermes, K.~Koh, P.~Koval,
S.~Lehtola, Z.~Li, J.~Liu, N.~Mardirossian, J.~D.~McClain,
M.~Motta, B.~Mussard, H.~Q.~Pham, A.~Pulkin, W.~Purwanto,
P.~J.~Robinson, E.~Ronca, E.~R.~Sayfutyarova, M.~Scheurer,
H.~F.~Schurkus, J.~E.~T.~Smith, C.~Sun, S.-N.~Sun, S.~Upadhyay,
L.~K.~Wagner, X.~Wang, A.~White, J.~D.~Whitfield, M.~J.~Williamson,
S.~Wouters, J.~Yang, J.~M.~Yu, T.~Zhu, T.~C.~Berkelbach,
S.~Sharma, A.~Y.~Sokolov, and G.~K.-L.~Chan,
``Recent developments in the \textsc{PySCF} program package,''
J.\ Chem.\ Phys.\ \textbf{153}, 024109 (2020).

\bibitem{Chung1997}
F.~R.~K.~Chung,
\emph{Spectral Graph Theory}
(American Mathematical Society, Providence, RI, 1997).

\bibitem{Fock1935}
V.~Fock,
``Zur Theorie des Wasserstoffatoms,''
Z.\ Phys.\ \textbf{98}, 145 (1935).

\bibitem{GeoVacPaper7}
J.~Loutey,
``The Dimensionless Vacuum: Recovering the Schr\"{o}dinger Equation
from Scale-Invariant Graph Topology,''
preprint (2026).

\bibitem{Marques2012}
M.~A.~L.~Marques, N.~T.~Maitra, F.~M.~S.~Nogueira, E.~K.~U.~Gross,
and A.~Rubio, \emph{Fundamentals of Time-Dependent Density Functional
Theory} (Springer, Berlin, 2012).

\bibitem{Li2020}
X.~Li, N.~Govind, C.~Isborn, A.~E.~DePrince, and K.~Lopata,
``Real-time time-dependent electronic structure theory,''
Chem.\ Rev.\ \textbf{120}, 9951 (2020).

\bibitem{Yabana1996}
K.~Yabana and G.~F.~Bertsch,
``Time-dependent local-density approximation in real time,''
Phys.\ Rev.\ B \textbf{54}, 4484 (1996).

\end{thebibliography}

\end{document}
